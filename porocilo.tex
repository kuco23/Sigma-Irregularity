\documentclass[ letterpaper, titlepage, fleqn]{article}

\usepackage[utf8]{inputenc}
\usepackage[slovene]{babel}
\usepackage[margin=60px]{geometry}
\usepackage{amsmath}
\usepackage{amssymb}
\usepackage{enumerate}
\setlength\parindent{0pt}

\begin{document}

\title{$\sigma$-irregularity vs. total $\sigma$-irregularity}
\author{Nejc Ševerkar \& Anja Trobec}
\date{\today}
\maketitle

\section{Uvod}
V projektni nalogi se bova ukvarjala z merjenjem iregularnosti enostavnih neusmerjenih grafov z
dvema na videz podobnima metodama, {\em $\sigma$-irregularity} in {\em total $\sigma$-irregularity},
definirani kot 
$$
\sigma(G) = \sum_{(u, v) \in E(G)}(d_u - d_v)^2 
\quad \text{in} \quad
\sigma_t(G) = \sum_{(u, v) \in V(G)}(d_u - d_v)^2
$$
Cilj naloge je maksimizacija razmerja, definiranega kot 
$$\sigma_r(G) = \frac{\sigma_t(G)}{\sigma(G)}$$
pri danem redu grafa $n \in \mathbb{N}$ in tako ugotoviti 
stopnjo naraščanja $\sigma_r(G)$, v odvisnosti od reda.

Ker nas grafi, za katere to razmerje ni definirano,
torej v primeru $\sigma(G) = 0$, ne zanimajo, definiramo $\sigma_r(G) = 0$.
To je natanko tedaj, kadar so vse komponenete grafa $G$ regularne.

\section{Osnovna Teorija}

\subsection{Maksimalna Stopnja Naraščanja}
Najprej razčistimo kaj je zgornja meja naraščanja $\sigma_r$ v odvisnosti
od reda grafa $G$. Ker velja
$$
\sigma_r(G) = \frac{\sigma_t(G)}{\sigma(G)} 
\leq \sigma_t(G)
= \sum_{(u, v) \in V(G)}(d_u - d_v)^2 
< \sum_{(u, b) \in V(G)}n^2
< n^2  n^2 = n^4
$$
Je red naraščanja $O(n^4)$.

\subsection{Nepovezani Grafi}
Ker sva opazila, da družina nepovezanih grafov doseže maksimalno stopnjo naraščanja,
se lahko po konstrukciji družine takšnih grafov $G_n$ za $\forall n \in \mathbb{N}$
osredotočimo samo na povezane grafe.
\\\\
Konstrukcije grafov $G_n$, za katere ima zaporedje $(\sigma_r(G_n))_n$ stopnjo 
naraščanja $O(n^4)$ je sledeča.
Vzamemo $n / 2$ vozlišč in iz njih konstruiramo poln graf, 
medtem, ko v preostalih $n /2$ vozliščih povežemo 3 vozlišča z dvema povezavama.
$$\sigma_r(G_{2n}) \geq \frac{n^2 \binom{n - 3}{2}}{2} = \frac{n^2 (n - 3)(n - 4)}{4} = \Theta(n^4)$$.
\\
S tem smo zaključili proučevanje nepovezanih grafov.

\section{Implementacija}
Problem bova reševala v Pythonu in si občasno pomagala s knjižnjico Networkx,
končna implementacija pa bo zaradi hitrosti verjetno napisana v jeziku c++.

\subsection{Metaheuristike}
Za optimalno vrednost $\sigma_r$ na grafih reda $n$ morava testirati vse neizomorfne
grafe tega reda, katerih je $\Omega(2^n)$. Če upoštevamo, da izračun $\sigma_r(G)$ 
zahteva $\Omega(n^2)$ operacij dobimo skupno časovno zahtevnost $\Omega(n^2 2^n)$.
Očitno je, zahtevnost predstavljala problem že za grafe reda $10$, 
torej bova morala poiskati alternativen pristop v obliki metaheurističnih algoritmov. \\

Ideja bo torej sistematično postopati po prostoru povezanih enostavnih grafov reda $n$ in 
tako iskati aproksimacijo grafa $G$, ki maksimizira vrednost $\sigma_r$ na tem prostoru.

Za učinkovito delovanje teh procesov, pa potrebujemo definirati ustrezno topologijo 
na prostoru, torej podati pojem bližine, saj jo zahteva večina heurističnih algoritmov.

To bova naredila v obliki zaporednega dodajanja ali odstranjevanja naključnih povezav v 
danem grafu, pri čimer morava paziti, da ohranjava povezanost grafa, 
torej z drugimi besedami, ne odstraniva mostov.\\
Za te namene je napisana knjižnjica, ki podajo podporo za izbiro teh povezav
in splošno generiranje naključnih povezanih grafov.


\subsection{Simulated Annealing}
Eden od algoritmov, ki naj bi rešil ta problem je {\em Simulated Annealing}, 
katerega implementacija je končana, a brez okolice, ki bi vrnila dobre rezultate.
Iz tega razloga sva poskusila implementacijo drugega algoritma

\subsection{Variable Neighborhood Search}
Trenutni algoritem, ki je v poteku implementacije je {\em Variable Neighborhood Search},
ki definira dve družini okolic, tj. globalno in lokalno, katere uporablja za iskanje optimalnega grafa.

\end{document}