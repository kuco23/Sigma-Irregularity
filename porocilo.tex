\documentclass[ letterpaper, titlepage, fleqn]{article}

\usepackage[utf8]{inputenc}
\usepackage[slovene]{babel}
\usepackage[margin=60px]{geometry}
\usepackage{amsmath}
\usepackage{amssymb}
\usepackage{enumerate}
\setlength\parindent{0pt}

\begin{document}

\title{$\sigma$-irregularity vs. total $\sigma$-irregularity}
\author{Nejc Ševerkar \& Anja Trobec}
\date{\today}
\maketitle

\section{Uvod}
V projektni nalogi se bova ukvarjala z merjenjem iregularnosti enostavnih neusmerjenih grafov z
dvema na videz podobnima metodama, {\em $\sigma$-irregularity} in {\em total $\sigma$-irregularity},
definiranima kot 
$$
\sigma(G) = \sum_{(u, v) \in E(G)}(d_u - d_v)^2 
\quad \text{in} \quad
\sigma_t(G) = \sum_{(u, v) \in V(G)}(d_u - d_v)^2
$$
Cilj naloge je maksimizacija razmerja, definiranega kot 
$$\sigma_r(G) = \frac{\sigma_t(G)}{\sigma(G)}$$
pri danem redu grafa $n \in \mathbb{N}$ in tako ugotoviti 
stopnjo naraščanja $\sigma_r(G)$, v odvisnosti od reda.

Ker nas grafi, za katere to razmerje ni definirano,
torej v primeru $\sigma(G) = 0$, ne zanimajo, definiramo $\sigma_r(G) = 0$.
To je natanko tedaj, kadar so vse komponenete grafa $G$ regularne.

\section{Osnovna Teorija}

\subsection{Maksimalna Stopnja Naraščanja}
Najprej razčistimo, kaj je zgornja meja naraščanja $\sigma_r$ v odvisnosti
od reda grafa $G$. Ker velja
$$
\sigma_r(G) = \frac{\sigma_t(G)}{\sigma(G)} 
\leq \sigma_t(G)
= \sum_{(u, v) \in V(G)}(d_u - d_v)^2 
< \sum_{(u, v) \in V(G)}n^2
< n^2  n^2 = n^4,
$$
je red naraščanja $O(n^4)$.

\subsection{Nepovezani Grafi}
Ker sva opazila, da družina nepovezanih grafov doseže maksimalno stopnjo naraščanja,
se lahko po konstrukciji družine takšnih grafov $G_n$ za $\forall n \in \mathbb{N}$
osredotočimo samo na povezane grafe.
\\\\
Konstrukcije grafov $G_n$, za katere ima zaporedje $(\sigma_r(G_n))_n$ stopnjo 
naraščanja $O(n^4)$ je sledeča.
Vzamemo $n / 2$ vozlišč in iz njih konstruiramo poln graf
medtem, ko v preostalih $n /2$ vozliščih povežemo 3 vozlišča z dvema povezavama.
Po kratkem premisleku lahko konstruiramo naslednjo neenakost.
$$\sigma_r(G_{2n}) > \frac{(n - 3)n \binom{n}{2}}{2} = \frac{(n - 3)n n(n - 1)}{4} = \Theta(n^4)$$.
\\
S tem smo zaključili proučevanje nepovezanih grafov.

\section{Ideja implementacije}
Problem bova reševala v Pythonu in si občasno pomagala s knjižnjico Networkx,
končna implementacija pa bo zaradi hitrosti verjetno napisana v jeziku c++.

%%%%%%%%%%%%%%%%%%%%%%%
\subsection{Metaheuristike}
Za optimalno vrednost $\sigma_r$ na grafih reda $n$ morava testirati vse neizomorfne
grafe tega reda, katerih je $\Omega(2^n)$. Če upoštevamo, da izračun $\sigma_r(G)$ 
zahteva $\Omega(n^2)$ operacij, dobimo skupno časovno zahtevnost $\Omega(n^2 2^n)$.
Očitno je zahtevnost predstavljala problem že za grafe reda $10$, 
torej bova morala poiskati alternativen pristop v obliki metaheurističnih algoritmov. \\

Ideja bo torej sistematično postopati po prostoru povezanih enostavnih grafov reda $n$ in 
tako iskati aproksimacijo grafa $G$, ki maksimizira vrednost $\sigma_r$ na tem prostoru.

Za učinkovito delovanje teh procesov pa potrebujemo definirati ustrezno topologijo 
na prostoru, torej podati pojem bližine, saj jo zahteva večina heurističnih algoritmov.

To bova naredila v obliki zaporednega dodajanja ali odstranjevanja naključnih povezav v 
danem grafu, pri čimer morava paziti, da ohranjava povezanost grafa, 
torej z drugimi besedami ne odstraniva mostov.\\
Za te namene je napisana knjižnjica, ki podajo podporo za izbiro teh povezav
in splošno generiranje naključnih povezanih grafov.

\subsection{Simulated Annealing}
Eden od algoritmov, ki naj bi rešil ta problem je {\em Simulated Annealing}, 
katerega implementacija je končana, a brez okolice, ki bi vrnila dobre rezultate.
Iz tega razloga sva poskusila implementacijo drugega algoritma

\section{Naloga}

\subsection{Majhni grafi}
Za majhne grafe sva na strani ** našla že generirane povezane neizomorfne grafe do stopnje 9.
Na njih sva poiskala maksimalne $\sigma_r(G)$, ki so zapisane v naslednji tabeli.
\begin{center}
    \begin{tabular}{ l  c  r }
      \hline
      $n$ & $\sigma_r(G)$ \\ \hline
      2 & 0 \\ \hline
      3 & -1 \\ \hline
      4 & 8 \\ \hline
      5 & 8 \\ \hline
      6 & 8 \\ \hline
      7 & 8 \\ \hline
      8 & 8 \\ \hline
      9 & 8 \\ \hline
      \hline
    \end{tabular}
  \end{center}

\subsection{Implementacija Simulated Annealing algoritma}

Algoritem sva implementirala kot prikazuje sledeča psevdokoda.

%\begin{small}
%    \begin{algorithmic}
%%    \Function{Simmulated_Annealing}{$n, nsim, alterLocal, alterGlobal$}
%    **insert ideja**
%    \EndFunction
%    
%    \end{algorithmic}
%    
%%    \end{small}
%\end{document}

Algoritem sprejme 3 argumente, ki so {\em $n$} število vozlišč in dve definiciji
okolic grafa {\em alterLocal}, {\em alterglobal}. 
Prva okolica definira majhne spremembe v grafu, druga pa definira nov graf,
ki ga algoritem konstruira, če predolgo ne naredi ustreznih sprememb.
Algoritem služi kot splošen model za iskanje maksimalnih vrednosti
$\sigma_r(G)$, specificiran z argumentoma, ki definirata ustrezne okolice.

Poglejmo si idejo implementacije okolic.
**insert psevdokoda** 

Lokalna okolica grafa je definirana s testiranjem parov števil naključno
odstranjenih in dodanih povezav. Iz tega vzorčenja izberemo optimalno 
spremembo.
Globalna okolica grafa pa sestavi nov graf kot zlepek polnega grafa in poti, 
saj se je takšna oblika izkazala za optimalno po testiranju grafov na
bolj splošni globalni okolici.

Po dobljenem optimumu se lotimo še popravljanja, ki nastopi v obliki
minimizacije $\sigma(G)$ tako da se osredotočimo na povezave, kjer
je razlika med $d_G(v)$ in $d_G(u)$ za neka $uv \in E(G)$ največja.

\subsection{Hipoteza}

Za optimalne grafe je dana hipoteza, da so krajišča njihovih povezav
vedno na absolutni razliki $1$.

Vsakemu grafu bova dodelila stopnjo neresničnosti izjave.
Ta stopnja bo definirana kot ...

\subsection{Distribucija stopenj vozlišč}
Narisala bova graf na katerem bosta za graf $G$ velikosti $n$ izpisani
vrednosti $\sigma_r(G)$ in $\max_{v \in V(G)}d_G(v)$ na katerem
se bo lepo videla korelacija.

Preveriti morava še, da je stopnja narašanja zaporedja
$(\sigma_r(G_n))_n$, kjer $G_n$ graf stopnje $n \in \mathbb{N}$
$O(n^2)$ in poiskati ustrezno konstanto $c \in mathbb{R}$, ki 
se naraščanju najbolj prilega.

To bova naredila tako, da bova za neko diskretizacijo intervalov
poiskala najboljše prileganje grafa funkcije $cn^p$ za 
$p \in [1, 4]$, $c$ pa bova dobila z metoda najmanjših kvadratov.
Tako bova ocenila naraščanje zaporedja in hkrati poiskala ustrezno
konstanto, ki pripada temu naraščanju.