\documentclass[ letterpaper, titlepage, fleqn]{article}

\usepackage[utf8]{inputenc}
\usepackage[slovene]{babel}
\usepackage[margin=60px]{geometry}
\usepackage{amsmath}
\usepackage{amssymb}
\usepackage{enumerate}
\usepackage[]{algorithm2e}
\setlength\parindent{0pt}

\begin{document}

\title{$\sigma$-irregularity vs. total $\sigma$-irregularity}
\author{Nejc Ševerkar \& Anja Trobec}
\date{\today}
\maketitle

\section{Uvod}
V projektni nalogi se bova ukvarjala z merjenjem iregularnosti enostavnih neusmerjenih grafov z
dvema na videz podobnima metodama, {\em $\sigma$-irregularity} in {\em total $\sigma$-irregularity},
definiranima kot 
$$
\sigma(G) = \sum_{(u, v) \in E(G)}(d_u - d_v)^2 
\quad \text{in} \quad
\sigma_t(G) = \sum_{(u, v) \in V(G)}(d_u - d_v)^2
$$
Cilj naloge je maksimizacija razmerja, definiranega kot 
$$\sigma_r(G) = \frac{\sigma_t(G)}{\sigma(G)}$$
pri danem redu grafa $n \in \mathbb{N}$ in tako ugotoviti 
stopnjo naraščanja $\sigma_r(G)$, v odvisnosti od reda.

Ker nas grafi, za katere to razmerje ni definirano,
torej v primeru $\sigma(G) = 0$, ne zanimajo, definiramo $\sigma_r(G) = 0$.
To je natanko tedaj, kadar so vse komponenete grafa $G$ regularne.

\section{Osnovna Teorija}

\subsection{Maksimalna Stopnja Naraščanja}
Ker bomo za analizo in primerjavo potrebovali zgornjo mejo stopnje naraščanja
zaporedja $(\sigma_r(G_n))_n$ v odvisnosti od reda $n$ si poglejmo 
naslednji račun.
$$
\sigma_r(G) = \frac{\sigma_t(G)}{\sigma(G)} 
\leq \sigma_t(G)
= \sum_{(u, v) \in V(G)}(d_u - d_v)^2 
< \sum_{(u, v) \in V(G)}n^2
< n^2  n^2 = n^4,
$$
Sledi, da je red naraščanja $O(n^4)$.

\subsection{Nepovezani Grafi}
Ker sva opazila, da družina nepovezanih grafov doseže maksimalno stopnjo naraščanja,
se lahko po konstrukciji družine takšnih grafov $G_n$ za $\forall n \in \mathbb{N}$
osredotočimo samo na povezane grafe.
\\\\
Konstrukcije grafov $G_n$, za katere ima zaporedje $(\sigma_r(G_n))_n$ stopnjo 
naraščanja $O(n^4)$ je sledeča.
Vzamemo $n / 2$ vozlišč in iz njih konstruiramo poln graf
medtem, ko v preostalih $n /2$ vozliščih povežemo 3 vozlišča z dvema povezavama.
Po kratkem premisleku lahko formuliramo naslednjo neenakost.
$$\sigma_r(G_{2n}) > \frac{(n - 3)n \binom{n}{2}}{2} = \frac{(n - 3)n n(n - 1)}{4} = \Theta(n^4)$$.
\\
S tem sva zaključila preučevanje nepovezanih grafov.

\section{Ideja Iskanja Optimalnih Grafov}
Problem bova reševala v Pythonu in si občasno pomagala s knjižnjico Networkx,
končna implementacija pa bo zaradi hitrosti verjetno napisana v jeziku c++.

\subsection{Metaheuristike}
Za optimalno vrednost $\sigma_r$ na grafih reda $n$ morava testirati vse neizomorfne
grafe tega reda, katerih je $\Omega(2^n)$. Če upoštevamo, da izračun $\sigma_r(G)$ 
zahteva $\Omega(n^2)$ operacij, dobimo skupno časovno zahtevnost $\Omega(n^2 2^n)$.
Očitno ta zahtevnost predstavlja problem že za grafe reda $10$, 
torej bova morala poiskati alternativni pristop v obliki metaheurističnih algoritmov. \\

Ideja bo torej sistematično postopati po prostoru povezanih enostavnih grafov reda $n$ in 
tako iskati aproksimacijo grafa $G$, ki maksimizira vrednost $\sigma_r$ na tem prostoru.

Za učinkovito delovanje teh procesov pa potrebujemo definirati ustrezno topologijo 
na prostoru, torej podati pojem bližine, saj jo zahteva večina heurističnih algoritmov.

To bova naredila v obliki zaporednega dodajanja ali odstranjevanja naključnih povezav v 
danem grafu, pri čimer morava paziti, da ohranjava povezanost grafa, 
torej z drugimi besedami ne odstraniva mostov.\\
Za te namene je napisana knjižnjica, ki poda podporo za izbiro teh povezav
in splošno generiranje naključnih povezanih grafov.

\subsection{Simulated Annealing}
Eden od algoritmov, ki naj bi rešil ta problem je {\em Simulated Annealing}, 
katerega implementacija je končana, a brez okolice, ki bi vrnila dobre rezultate.
Iz tega razloga sva poskusila implementacijo drugega algoritma.

\section{Naloga}

\subsection{Majhni grafi}
Za majhne grafe sva na strani ** našla že generirane povezane neizomorfne grafe do stopnje 9.
Na njih sva poiskala maksimalne $\sigma_r(G)$, ki so zapisane v naslednji tabeli.

\begin{center}
    \begin{tabular}{|  l | c | }
      \hline
      $n$ & $\sigma_r(G)$ \\ \hline
      2 & 0 \\ 
      3 & 1 \\ 
      4 & -1 \\ 
      5 & 3 \\ 
      6 & 5 \\ 
      7 & 13 \\ 
      8 & 19 \\ 
      9 & 14.5 \\ 
      \hline
    \end{tabular}
  \end{center}

Grafov na večjemu številu vozlišč je preveč za posamično analizo, torej
se lotimo implementacije {\em Simulated Annealing} algoritma.

\subsection{Večji grafi}

Med večjimi grafi bomo optimume iskali s pomočjo algoritma
{\em Simulated Annealing}. Ta sprejme 2 argumenta,
{\em $n$} število vozlišč in definicijo
okolice grafa {\em alterLocal}.
Okolice so lahko poljubne in se delijo na lokalne in globalne.

Lokalna okolica grafa je definirana s testiranjem parov števil naključno
odstranjenih in dodanih povezav. Iz tega vzorčenja izberemo optimalno 
spremembo.

Globalna okolica grafa pa sestavi nov graf kot zlepek polnega grafa in poti, 
saj se je takšna oblika izkazala za optimalno po testiranju grafov na
bolj splošni globalni okolici.

Po dobljenem optimumu se lotimo še popravljanja, ki nastopi v obliki
minimizacije $\sigma(G)$ tako da se osredotočimo na povezave, kjer
je razlika med $d_G(v)$ in $d_G(u)$ za neka $uv \in E(G)$ največja.

\subsection{Hipoteza}

Za optimalne grafe je dana hipoteza, da so krajišča njihovih povezav
vedno na absolutni razliki $1$.

To bova preverila tako, da bova izračunala število vozlišč za katera
to ne velja pri vsakem optimalnem grafu in ta prikazala z grafom.

\subsection{Distribucija stopenj vozlišč}
Narisala bova graf na katerem bosta za graf $G$ velikosti $n$ izpisani
vrednosti $\sigma_r(G)$ in $\max_{v \in V(G)}d_G(v)$ na katerem
se bo lepo videla korelacija.

Preveriti morava še, da je stopnja narašanja zaporedja
$(\sigma_r(G_n))_n$, kjer $G_n$ graf stopnje $n \in \mathbb{N}$,
$O(n^2)$ in poiskati ustrezno konstanto $c \in \mathbb{R}$, ki 
se naraščanju najbolj prilega.
\\\\
Predpostavimo, da smo optimum izračunali za $n$ grafov in
označimo $X(n, p) = (1^p, 2^p, \dots, n^p)^T \in \mathbb{R}^{nx1}$ 
in $a = (a_1, a_2, \dots, a_n)^T$ vektor, kjer $a_i$ predstavlja
dobljeni optimalni $\sigma_r(G_i)$ za graf $G_i$ na $i$ vozliščih.
Problem pri danem $p \in \mathbb{R}^{+}$ zahteva rešitev linearnega sistema 
$$cX(n, p) = a$$.
Ta sistem seveda ne bo rešljiv, iskanja aproksimacije pa se bomo
lotili z metodo najmanjših kvadratov, torej minimum $2$-norme
razlike obeh strani bo dosežen pri
$$c = ||X(n, p)||_{2}^2 / <a, X(n, p)>$$.

Najboljšo aproksimacijo naraščanja $\sigma_r$ bomo poiskali 
z diskretizacijo $D \subset [0, 4]$ in za $\forall p \in D$ 
izračunali prej definirani $c$, na koncu pa izbrali tisto kombinacijo
$(c, p)$, za katero je $||cX(n, p) - a||_{2}$ najmanjši.

\subsection{Časovna zahtevnost}

\end{document}
